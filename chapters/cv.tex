\pagestyle{backmatter}

\chapter{Curriculum Vitae}
\label{ch: cv}

Yongbin Wei was born on April 16th, 1990, in Zunyi, China. He graduated high school in 2008 at the Zunyi Hang-Tian Middle School (ranked top 0.5\% in the National College Entrance Examination in Guizhou province). After that, he started his bachelor's degree in \textit{Electronic and Information Engineering} at the Beihang University (BUAA), Beijing. He received his bachelor's degree in 2012 and then started his master's study in \textit{Computer Science and Application} at the Beijing Normal University, Beijing. During his studies, he joined the Laboratory of Imaging Connectomics at the National Key Laboratory of Cognitive Neuroscience and Learning led by Prof. Dr. Yong He. He did researches about the functional connectomics and Alzheimer's Disease, and he became interested in applying the connectomic approaches to understand brain disorders. \\

In 2015, Yongbin moved to Utrecht, the Netherlands and started his PhD in the Dutch Connectome Lab (DCL) at the Department of Psychiatry, University Medical Center Utrecht, under supervision of Prof. Dr. René Kahn and Prof. Dr. Martijn van den Heuvel. With the relocation of the DCL in 2018, he moved to the Department of Complex Trait Genetics, Center for Neurogenomics and Cognitive Research, VU Amsterdam, Amsterdam and continued his PhD under supervision of Prof. Dr. Martijn van den Heuvel and Prof. Dr. Danielle Posthuma. During his PhD, he focused on exploring the neurobiological mechanism underlying the human brain connectome and improving our understanding of the connectome-based pathology in psychiatric diseases. Yongbin aims to continue his research journey as a network neuroscientist, implementing multi-scale, multi-species connectomic data and advanced analytical methodology to achieve a better understanding of our brain.










   


\pagestyle{backmatter}


\chapter{Curriculum vitae}
\label{ch: cv}

Yongbin Wei was born on April 16th, 1990, in Zunyi, Guizhou, China. He graduated from high school in 2008 at the Zunyi Hang-Tian Middle School (ranked top 0.5\% in the National College Entrance Examination in Guizhou province). After that he started his bachelor's study in Electronic and Information Engineering at the Beihang University, Beijing. He received his bachelor's degree in 2012 and then started his master's study in Computer Science and Application at the Beijing Normal University, Beijing. During his studies, he joined the Laboratory of Imaging Connectomics at the National Key Laboratory of Cognitive Neuroscience and Learning led by Prof. Dr. Yong He. Yongbin did researches on the functional connectomics and Alzheimer's Disease, and he became interested in applying network neuroscience approaches to understand brain diseases. \\

In 2015, Yongbin moved to Utrecht, the Netherlands and started his PhD study in the Dutch Connectome Lab (DCL) at the Department of Psychiatry, University Medical Center Utrecht, under the supervision of Prof. Dr. René Kahn and Dr. Martijn P. van den Heuvel. In 2018, he moved to the Department of Complex Trait Genetics, Center for Neurogenomics and Cognitive Research, VU Amsterdam and continued his PhD under the supervision of Prof. Dr. Martijn P. van den Heuvel and Prof. Dr. Danielle Posthuma. During his PhD, he focused on exploring the neurobiological mechanism underlying the human brain connectome and improving our understanding of the connectome-based neuropathology in psychiatric diseases. Yongbin aims to continue his research journey as a network neuroscientist, implementing multi-modal, multi-scale, multi-species neuroscience data and advanced analytical methodology to achieve a better understanding of the brain.

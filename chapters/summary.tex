\pagestyle{MyStyle}

\begin{refsection}

\chapter{Summary and general discussion}
\label{ch:summary}

\newpage

In this chapter, I will first summarize the main findings presented in this thesis, followed by a general discussion with a comprehensive literature review. The discussion is categorized into two parts: 1) evidence for multiscale wiring principles of the human brain connectome is incorporated, and 2) observations in the context of multiscale neuropathology of schizophrenia are revisited. Furthermore, limitations and future directions in the field of multiscale connectomics will be deliberated.

\section*{Summary}
Integrating the genetic, histological, and brain imaging data have provided a rich body of evidence for the relationship across multiscale brain connectome organization (Scholtens and van den Heuvel, 2018; van den Heuvel et al., 2019a). Studies on non-human mammalian species have shown that the similarity of the microscale cortical cytoarchitecture is associated with the formation of macroscale cortico-cortical connectivity (Barbas, 2015; Beul et al., 2015; Beul et al., 2017). Chapter 2 provides further evidence for such a micro-macro association by combining the state-of-the-art ultrahigh-resolution BigBrain dataset and the macroscale connectome of the human brain. The laminar cytoarchitectonic profile of human cortical regions was delineated using the BigBrain dataset, and the connectome was reconstructed using diffusion weighted imaging (DWI) data based on homologous cortical regions. Bridging the two ends of scales showed that the cytoarchitectonic profiles were more similar between interconnected cortical regions as compared to non-connected regions. The level of the cytoarchitectonic profile similarity was positively correlated with the connectivity strength, suggesting that cortical regions with more similar cytoarchitecture tend to be connected by stronger white matter connections. These results thus confirm one of the important wiring principles of the connectome – the microscale cortical cytoarchitectonic similarity shapes the macroscale brain connectome organization – to be evolutionarily conserved across multiple species with no exception in the human. 

In Chapter 3, I further demonstrate the wiring principles of brain functional networks from the perspective of evolutionary genetics. In the human brain, there are several functional networks, such as the frontoparietal network (FPN), salience network (SN), and default-mode network (DMN), supporting higher-order brain functions that differentiate human from other intelligent evolutionary relatives (Buckner and Krienen, 2013). We thus hypothesized that brain regions of these higher-order cognitive networks were largely expanded in human evolution and this expansion was influenced by the underlying genetic differences between human and other non-human primates. To test this hypothesis, the cortical ribbon of humans and chimpanzees was first reconstructed using magnetic resonance imaging (MRI) data, and surface area of homologous cortical regions was compared between humans and chimpanzees. This comparison showed the highest cortical expansion in regions of higher-order cognitive networks, such as the FPN and DMN, in the human brain. Next, the pattern of brain expansion was linked to the gene expression pattern of the human-accelerated genes (HAR genes) in the human brain. Brain regions with more expansion in human evolution were found to demonstrate higher expression levels of HAR genes, with regions of the DMN showing the highest level of HAR gene expression. Comparative gene expression analysis further showed that HAR genes were differentially more expressed in higher-order cognitive networks in humans compared to chimpanzees and macaques. These findings together suggest that the upregulated expression of HAR genes may have played a role in the large expansion of cognitive functional networks during human brain evolution. Furthermore, chapter 3 identified a set of genes with specifically over-expression in the DMN and noted this gene set to be significantly overrepresented in HAR genes, and to be involved in synapse and dendrite formation. HAR genes and DMN genes show significant associations with individual variations in intelligence, sociability, and mental conditions such as schizophrenia and autism in today’s population. These findings highlight the potential role of HAR genes in shaping the higher-order cognitive functional networks in the human brain, and ultimately influencing cognitions and behaviors, and mental disorders of humans.

Schizophrenia is a mental disorder that is characterized by brain disconnectivity, in particular for connections linking highly connected rich-club regions. However, due to confounding factors such as prior therapeutic exposure and the potential influence of chronicity, it remained unclear to what extent the rich-club disconnectivity reflects the pathophysiology inherent to the nature of schizophrenia. Chapter 4 thus studied connectome abnormalities in first-episode, medication-naïve schizophrenia patients whose medical history is short and therapy is absent. DWI data and resting-state fMRI data from two independent samples were collected, including the principal dataset of 42 medication-naïve, previously untreated patients and 48 healthy controls, and the replication dataset of 39 first-episode patients (10 untreated patients) and 66 healthy controls. The connectome was reconstructed and the rich club organization was compared between patients and controls. We showed that the rich club organization was significantly disrupted in medication-naïve schizophrenia patients as compared to healthy controls, with rich club connection strength decreased in patients. The coupling between structural connectivity and functional connectivity among rich club regions was also decreased in medication-naïve schizophrenia patients. Using the replication dataset revealed similar results. These findings suggest that the disruption of rich club organization and functional dynamics may reflect an early feature of schizophrenia pathophysiology.

Macroscale brain disconnectivity in schizophrenia has been noted to be associated with microscale neuropathology (van den Heuvel et al., 2019a). Chapter 5 provides novel evidence for this cross-scale association by using in vivo magnetization transfer imaging (MTI), which provides an indirect measurement of brain microstructure, such as myelination (Whitaker et al., 2016). MTI data and DWI data from 78 schizophrenia patients and 93 healthy controls were collected to compute the magnetization transfer ratio (MTR) and to reconstruct the brain connectome, respectively. Significant MTR reductions were observed in prefrontal cortical regions in schizophrenia patients as compared to controls,  including bilateral rostral middle frontal areas, and right pars orbitalis and frontal pole. This finding confirms the prefrontal disruption of brain microstructure. Moreover, the cortical pattern of MTR reduction was observed to be associated with the pattern of macroscale dysconnectivity in schizophrenia, implicating regions with more myelination reduction to have more connectivity disruptions at the macroscale. This study thus provide empirical evidence for the prefrontal neuropathology in schizophrenia and suggest this microscale deficits to be associated with macroscale connectome abnormalities.

\section*{General discussion}
\subsection*{Multiscale wiring principles of the human brain connectome}
The major goal of this thesis is to provide new insights into the wiring principles of the human brain connectome, namely, the rules governing the formation of macroscale structural and functional brain connectivity. Emerging evidence across a wide range of species has indicated two evolutionarily conserved topological principles of the connectome: the community structure and the hub/rich-club organization (van den Heuvel et al., 2016a). The community structure at the macroscale describes that spatially or functionally close brain regions are preferentially connected to each other to form distinct communities or functional networks in relation to different cognitive domains (Smith et al., 2009; Sporns and Betzel, 2016; Wei et al., 2017). Hubs/rich-club brain regions are suggested to connect the distributed communities via long-range, more costly connections to enable efficient neural information integration within the whole brain (van den Heuvel et al., 2012; Senden et al., 2014). Taken together, network communities and hubs/rich-club display a trade-off between minimizing the cost of neural resources and maximizing the efficiency of information communication (Bullmore and Sporns, 2012; van den Heuvel and Sporns, 2019). This thesis discusses and extends these two topological motifs in the context of the microscale brain structure to gain a deeper understanding of wiring principles of the human brain connectome.

Our results in Chapter 2 support the notion that the microscale cytoarchitecture of cortical regions plays a role in shaping the macroscale brain connectivity. Neuroanatomical studies have demonstrated varied cytoarchitectonic features across the cortex, such as the number and thickness of cortical layers, degree of visibility of layer IV, cell packing density, et cetera. (Brodmann, 1909; von Economo and Koskinas, 1925; Zilles and Amunts, 2015). These cytoarchitectonic features are claimed to follow a system-level gradient from the primary sensory/motor to the association and eventually to the limbic system (Zilles and Amunts, 2015; Paquola et al., 2019). Studies combining information of microscale cytoarchitecture and macroscale anatomical connectivity have indicated that the cytoarchitectonic similarity between brain regions could predict the existence of their interconnections at the macroscale (Barbas, 2015). This hypothesis, known as “structural model” (Barbas, 2015), is supported by evidence showing that brain regions with more similar cytoarchitecture are more likely to be connected by white matter fibers in a wide range of species, such as mouse (Goulas et al., 2017), cat (Beul et al., 2015), and macaque (Hilgetag and Grant, 2010; Beul et al., 2017). Our observation of the associated cytoarchitectonic profile similarity and cortico-cortical connectivity in Chapter 2 further suggests the “structural model” to be evolutionarily conserved in the human brain (Wei et al., 2019a). This is also in line with results showing the similarity of cytoarchitecture and myeloarchitecture between brain regions to be associated with the macroscale functional network architecture in the human brain (Hunt et al., 2016; Huntenburg et al., 2017; Paquola et al., 2019), indicating the similarity of cellular-level brain structure as a underlying neural basis for the formation of brain functional networks.

Such a “similar-prefer-similar” principle of the connectome organization may be explained from several aspects. First, the developmental trajectory of the brain cytoarchitecture and brain connectivity may determine the association between microscale cytoarchitectonic similarity and macroscale brain connectivity. Briefly, brain regions with similar cytoarchitecture are likely to develop within a similar time window that overlaps in the time of brain connectivity formation (Barbas, 2015; Hilgetag et al., 2016; Beul et al., 2017). Moreover, the associated cytoarchitectonic similarity and macroscale connectivity may be driven by the underlying transcriptional signatures. Genetic findings have shown that brain regions with more similar gene expression profiles are preferentially connected by macroscale white matter connections (French and Pavlidis, 2011). The highest similarity of gene expression profiles corresponds to connections between rich-club brain regions, with the second highest similarity for connections between rich-club and non-rich-club regions, and the lowest similarity for connections between non-rich-club regions (Fulcher and Fornito, 2016). The coupled transcriptional profiles of genes also recapitulate patterns of resting-state brain functional networks (Richiardi et al., 2015). Specifically examining a subset of genes enriched in the supragranular layer of the human cortex have shown that brain regions with similar transcription profiles of these genes are tightly connected at the macroscale both structurally (Romero-Garcia et al., 2018) and functionally (Krienen et al., 2016). These findings together suggest that brain regions with similar microscale molecular and cellular architecture are likely to be both structurally and functionally connected at the macroscale.

The association between the microscale brain structure and the macroscale connectome is further supported by Chapter 3, which suggests an over-expression of human-accelerated  genes (HAR genes) in brain regions of higher-order cognitive networks. HAR genes are known to function as neuronal enhancers and play a pivotal role in neurogenesis (Ryu et al., 2018). Up-regulated expression of HAR genes in regions of cognitive functional networks thus may bring these brain regions more complex neuronal structure (Elston, 2003). More complex neuronal structure, such as higher dendritic spine density of layer III pyramidal neurons, has been linked to more white matter connections at the macroscale (Scholtens et al., 2014; van den Heuvel et al., 2016b). Therefore, there might be a linkage across multiscale brain structures: the microscale molecular and cellular structure, and the macroscale brain connectivity. This linkage is also supported by findings showing that genes related to the functional activity of the DMN are those more expressed in neurons (Wang et al., 2015) and are involved in the formation of neuronal projections (Wei et al., 2019b). Genes over-expressed in superior and lateral association cortex that are largely involved in long-range inter-module functional connectivity are also significantly enriched in supragranular layers of human cortex that is important for the wiring of cortico-cortical connectivity (Vertes et al., 2016). Furthermore, as HAR genes correspond to genome regions conserved in other primates but strikingly differentiated in human, results in Chapter 3 point to an evolutionary adaption for the observed multiscale association.

Chapter 3 shows that the organization of human brain functional networks undergoes selection pressure during human evolution. Despite humans and non-human primates share similar patterns of functional networks, such as the DMN (Mantini et al., 2011; Barks et al., 2015), association brain regions involved in higher-order cognitive networks have been largely expanded in the human brain compared to the brain of non-human primates (Hill et al., 2010; Donahue et al., 2018). White matter connections linking these association areas in the human brain also overrepresent with connections that are uniquely observed in humans but not in chimpanzees (Ardesch et al., 2019). Such differentiations in macroscale brain structure are associated with the over-expression of HAR genes of the human genome [Chapter 3]. These evolutionary changes may be related to the observations that humans have a longer period of neuronal progenitor expansion compared to chimpanzees and macaques (Otani et al., 2016). The differentiated neuronal progenitor expansion may further relate to more complex neuronal structure in association regions of higher-order cognitive networks (Elston et al., 2001) and ultimately higher computational capacity to support complicated cognitive abilities of humans (Bianchi et al., 2013; Goriounova et al., 2018; Reardon et al., 2018). However, these evolutionary changes may also make higher-order cognitive networks more vulnerable to psychiatric conditions (Wei et al., 2019b), in particular schizophrenia (van den Heuvel et al., 2019b).

\subsection*{Multiscale neuropathology in schizophrenia}
Schizophrenia has long been suggested as a disorder of brain dysconnectivity (Friston, 1998; Stephan et al., 2009; Fitzsimmons et al., 2013). While disruptions of structural connectivity in schizophrenia are widespread within the entire brain, the dysconnectivity pattern topologically converges to brain hub/rich-club regions (Klauser et al., 2017). Connections between hub/rich-club regions consistently show reductions in the connectivity strength in schizophrenia (van den Heuvel et al., 2013; Griffa et al., 2015; Klauser et al., 2017). Similar results have been revealed in first-episode, medication-naïve schizophrenia patients [Chapter 4] (Cui et al., 2019), suggesting rich club abnormality in schizophrenia as a white matter pathology that is independent to the drug usage. Intriguingly, rich club disruptions have been found in unaffected siblings (Collin et al., 2014) and young offspring of schizophrenia patients (Collin et al., 2017), further suggesting the potential genetic substrates of the disconnectivity pattern in schizophrenia.

The genome-wide association studies (GWAS) in schizophrenia have revealed a complex genetic component of schizophrenia, showing that multiple genomic loci, for example loci in DRD2 and several genes involved in glutamatergic neurotransmission,  together contribute to the etiology of schizophrenia (Schizophrenia Working Group of the Psychiatric Genomics, 2014; Li et al., 2017). Cross-scale examinations have shown that genetic risks of schizophrenia are associated with macroscale brain structure and function. Polygenic risks of schizophrenia have been observed to be associated with individual variability in gray matter and white matter volume (Terwisscha van Scheltinga et al., 2013), cortical thickness (Alnaes et al., 2019), structural connectivity (Alloza et al., 2018), functional connectivity (Wang et al., 2017), and brain activity when processing working memory tasks (Kauppi et al., 2015). Moreover, examining cortical transcription profiles in healthy brains shows that cortical regions with higher expression levels of schizophrenia risk genes have more severe disruptions in macroscale white matter connectivity in schizophrenia (Romme et al., 2017). This further suggests that brain processes at multiple scales are not independent and schizophrenia-related brain connectivity alterations at the macroscale may parallel molecular changes at the microscale.

Extended evidence for the cross-scale pathological association in schizophrenia come from studies examining disease-related cellular abnormalities. Investigating cytoarchitecture alone have indicated the reduced dendritic spine density (Garey et al., 1998; Glantz and Lewis, 2000; Kolluri et al., 2005) and soma size (Pierri et al., 2001) of layer III pyramidal neurons in prefrontal cortex in schizophrenia patients. Studies relevant to myeloarchitecture have shown reduced density of oligodendrocytes, which are responsible for forming myelin sheath around axons, in prefrontal regions in schizophrenia (Uranova et al., 2004; Uranova et al., 2011). Above microscale changes are very likely to be related to the macroscale brain alterations in schizophrenia, with evidence showing a strong correlation between the alteration of dendritic spine density of layer III pyramidal neurons and the macroscale white matter connectivity changes in schizophrenia (van den Heuvel et al., 2016b). Moreover, the disruptions in cortical myelination, as discussed in Chapter 5 (Wei et al., 2018), has also been suggested to be related to white matter connectivity changes in schizophrenia (Cassoli et al., 2015). Taken together, linking multiscale brain information in schizophrenia provide a more comprehensive understanding for the pathway of schizophrenia, which are important for understanding the etiology of the disease.

\subsection*{Methodology considerations and future directions}
Several methodological considerations need to be remarked when interpreting results in this thesis. Many of these considerations also point to opportunities for future studies. 

Studies in Chapter 2 and 3 bridge microscale brain cytoarchitecture and transcriptomics to macroscale brain connectome based on data resources acquired across different individuals, such as the BigBrain data (Amunts et al., 2013), Allen Human Brain Atlas (Hawrylycz et al., 2012), psychENCODE dataset (Sousa et al., 2017), and MRI data from the Human Connectome Project (Van Essen et al., 2013). The cross-scale association is mainly built upon the relationship of group-level spatial profiles across brain regions rather than across individuals, such that the individual variation of multiscale features of the human brain is inherently overlooked. Studying individual variation is quite challenging at the moment as it requires different scales of data from the same subjects. Therefore, new datasets delineating multiscale brain structure and function from the same subjects would be of great value. Psychiatric brain banks (Deep-Soboslay et al., 2011) that collect multiscale data from brain tissues of psychiatric patients may give the opportunity for connecting individual variations across different scales. Moreover, the advance of image technology and analytic methods provides promising approaches to assess microscale brain structure. For example, the contrast of T1- and T2-weighted MRI (Glasser and Van Essen, 2011) and MTR imaging [Chapter 5] has been widely used to measure the extent of cortical myelination (Heath et al., 2018). Ultra-high-field quantitative MRI has the potential to assess the in vivo laminar structure (Trampel et al., 2019) and molecular composition (Filo et al., 2019) of the human cortex. These developments in imaging techniques may make it feasible to collect in vivo multiscale data from larger samples of human subjects and to bridge individual variations of microscale cytoarchitecture and myeloarchitecture and macroscale connectome.

An additional consideration is that the sample size of some of the examined datasets in this thesis is relatively small due to the highly complex and expensive processes in data acquisition. The small sample size make it challenging for validation and generalization of the results. In this thesis, we put much efforts on validation. For example, we use the Von Economo - Koskinas atlas to show the validity of BigBrain profiles examined in Chapter 2 and replicate the major findings in Chapter 3 by using both the AHBA dataset and psychENCODE dataset. However, increasing data samples to validate results based on separate sets of samples with similar configurations, as similar as what Chapter 4 does, would still be of great importance to generalize the discussed multiscale associations.  

It is noteworthy that the correlation analysis used in Chapter 2, 3, 5 to cross-reference multiscale brain features do not provide information about the causal interactions across scales. One possibility is that the macroscale connectome organization is a result of the wiring of microscale neural circuits, which are determined by gene transcription (van den Heuvel et al., 2019a). This step-by-step influence may also reflect the mechanism in schizophrenia, proposing the disconnectivity at the macroscale is a result of abnormalities in microscale neuronal structure and genotypic changes of schizophrenia risk genes. Alternatively, the macroscale connectome organization may also play a role in changing the microscale features. For example, the organization of macroscale brain functional networks has been proposed to be able to guide the spread of cellular pathology in neurological disorders (Buckner et al., 2009; Seeley et al., 2009; Zhou et al., 2012). Furthermore, the observed association may reflect a more complex bidirectional interactions between different scales. Brain features across different scales influence each other and dynamically change during the development and conversely the course of disease (Bullmore et al., 1997, Catani and ffytche, 2005). Future studies making modifications at the microscale brain structure and then examining connectome alterations at the macroscale would be of great help to figure out the certain mechanism. Longitudinal studies examining the multiscale changes during the development or the course of the disease may also provide meaning information about the causality of cross-scale interaction.

\subsection*{Conclusions}
This thesis examines the association between the macroscale brain connectome organization and the microscale cortical cytoarchitecture and gene transcription. It also provide evidence for the macroscale disconnectivity in schizophrenia and its relationship to the microscale cellular pathology of the disease. Binding genes, neurons, and the macroscale connectome, our results provide novel insights into the wiring principles of the human brain connectome and the disconnecitvity pattern in schizophrenia. Integrating information of the brain structure and function from multiple scales pave a new avenue to disentangle the complex neurobiological mechanisms of human’s cognitive abilities and psychiatric disorders.



\end{refsection}
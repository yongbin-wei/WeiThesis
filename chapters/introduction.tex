\pagestyle{MyStyle}

\chapter[Introduction]{Introduction}
\label{ch:intro}

\vspace{3cm}

\begin{refsection}
\newpage
\section*{The brain connectome}
The brain is the most sophisticated organ in the human body. The adult human brain consists of approximately 100 billion neurons and 150 trillion synapses, consuming 20\% of the total body energy by taking only 2\% of the total body mass (1). Through axonal projections and synapses, neurons are wired into neural circuits, which interconnect to one another to form the large-scale brain network, namely the connectome (2). Populations of neurons are embedded in continuous brain areas, and axonal projections are bounded into white matter fiber bundles that interconnect brain areas, forming the macroscale brain connectome (2, 3). The macroscale brain connectome describes a highly efficient network that enables both functional specialization and integration within the whole brain (4). It facilitates a rich body of functional dynamics and interactions that underlie human cognition and behavior, evidenced by associations between the connectome organization and cognitive processing in humans (5-7) and brain dysfunctions in mental disorders (8, 9). Together, the brain connectome provides an invaluable model for understanding the neurobiological mechanism of human’s cognitive abilities and the neuropathology of brain disorders.

\section*{Aim of this thesis}
One of the major challenges in terms of the connectomics is to understand the neurobiological wiring principle of the human brain connectome. In other words, what shapes the emergence of brain connectivity and brain network architecture? This thesis aims to answer this question by integrating multiscale data from brain transcriptomics, histology, and multi-modal magnetic resonance imaging (MRI), exploring the genetic and cytoarchitectonic underpinnings of the human brain connectome. Furthermore, this thesis aims to examine multimodal connectome changes in schizophrenia to provide insights into the neuropathology of this disease. We expect our integrative approaches to be useful in gaining a better knowledge of the wiring principle of the human brain connectome, and ultimately, a deeper understanding of human cognition and behavior and brain disorders.

\section*{General background}
\subsection*{Topological characteristics of brain network}
The macroscale brain connectome is a complex network composed of comprehensive sets of connections between brain areas. Mathematically, a network, or a graph, is defined as a set of nodes and edges. With respect to the human brain network, nodes are usually defined according to parcellation of homogenous and unique brain regions, which are established based on cortical cytoarchitecture (10), gyrification (11), connectivity profiles (12), or a combination of multiple modalities (13). Edges can be defined as interareal structural and functional connectivity among brain areas. Structural connectivity in the human brain commonly reflects white matter bundles inferred from diffusion MRI (dMRI) (14). Functional connectivity in the human brain is characterized as the statistical dependency of brain activity (15), such as the synchrony between blood oxygenation level-dependent (BOLD) signals derived from resting-state functional MRI (fMRI). Alternative definitions of brain connectivity, including but not limited to structural covariance (the synchrony of cortical morphology across population) (16) and effective connectivity (the causal relationship between brain activity) (15), are possible but beyond the scope of this thesis. Instead, we focus on the most fundamental attributes of the structural and functional brain networks: the capability of promoting the segregation and integration of neural information (17).

Neural information segregation complies with the long-stand argument of brain functional specialization, with evidence from a wide range of anatomical, physiological, and neuroimaging studies showing different brain areas to be specialized for different functions (18). Connectome analyses further add to this theory by demonstrating the architecture of communities in functional human brain networks (19) -- network nodes within a community are way more densely connected to each other in contrast to being connected to nodes outside the community. The observed sets of communities in the functional human brain network coincide with the distributed “resting-state networks” (RSNs) that are identified from independent component analysis (ICA) on whole-brain resting-state fMRI data (20). Both functional communities and ICA-derived RSNs are related to functional systems in specific behavioral and cognitive domains  (21, 22), ranging from the default mode (DMN) and frontoparietal control networks to the dorsal attention, ventral attention, sensorimotor, and visual networks (23, 24). These spatially-distributed functional systems are supported by anatomical connections (25) and are globally integrated via pivotal “hub” areas and their connections to facilitates complex brain functions (26).

Neural information integration is argued to be served by a specific set of brain hub regions (27). Hub regions of the structural brain network represent nodes that play a central role in brain network communication, and they are mostly located in multimodal association areas, such as the precuneus, insular cortex, superior frontal cortex, temporal cortex, and lateral parietal cortex (27). Studies further suggest that hub regions are not only highly connected to the rest of the brain, but also tightly connected among each other, forming a “rich club”  (28, 29). Rich club areas and their connections enable short communication pathways among peripheral areas, thus promote network efficiency with a trade-off against high wiring cost of connection among distributed hub regions (30). These central areas are also flexible over time, indicative of the capability of dynamically switching brain states from more localized processing (i.e., segregation) to more globalized synchronization (i.e., integration) (31, 32). Such flexibility enables the brain to adapt to changing demands (33) and facilitates faster and more accurate cognitive performance (32). 

Together, functional segregation and integration are two fundamental organizing rules of the human brain network. These two aspects play a key role in complex human brain dynamics underlying cognition and behavior. One major purpose of the current thesis is to understand the origin and emergence of such an organization of the human brain network.

\subsection*{Neurobiological basis of brain networks}
In recent years, studies have aimed to bridge microscale cellular and molecular structure to macroscale brain network organization to examine the neurobiological mechanisms of the emergence of brain network (34, 35). In the brain, cortical areas show a large variety of cellular composition (36, 37), with for example larger and more spinous pyramidal cells observed in associative prefrontal cortex compared to primary visual regions (e.g., primary visual cortex) (38, 39). Extending this cytoarchitectonic variation, studies have further proposed that the inter-areal cytoarchitectonic differentiation plays a key role in shaping cortico-cortical connections (40). Cortical regions with similar cytoarchitecture, which may indicate similar functional profiles, are more likely to be anatomically connected (40-44). Moreover, cortical cytoarchitecture is also associated with properties of functional segregation and integration in the brain. Paralimbic regions with an undifferentiated laminar structure, along with absent or disrupted cortical layer IV, are prominently involved in between-network connectivity across RSNs (45). Densely-connected rich-club regions show large basal dendritic tree size, spine density, and neuron soma size of layer III pyramidal neurons (46-48), suggestive of the high-level neuronal complexity, that maybe be related to high-level neural processing capacity, underlying macroscale neural information integration. 

Extending the relationship between brain connectome and cytoarchitecture, studies further demonstrated the brain network organization to be associated with the underlying genetic materials. Studies combining gene expression and brain connectivity in the rodent brain first showed that the existence of anatomical connectivity could be predicted by gene expression levels (49) and connected regions tended to show more similar expression profiles (50). Genes that are mostly correlated with anatomical connectivity were enriched for gene sets involved in neuronal development and axon guidance (50). Another study examining the mouse brain suggested that rich club regions showed the highest level of gene expression coupling, a phenomenon driven by genes regulating the oxidative synthesis and metabolism of ATP that provide the energetic substrate for neuronal communication (51). Furthermore, the topology of functional networks in the human brain were found to relate to co-expression of genes that are enriched for ion channels (52) and that are preferentially expressed in human supragranular cortical layers (53), with regions within the same functional system showing more similar transcriptional profiles. More specifically, resting-state brain activity of the default-mode network (DMN) show significant correlation with genes that also display greater expressions in neurons (54). These findings implicate that macroscale brain connectome and microscale cytoarchitecture and gene expression are tightly coupled to each other to form an integrative system that support our brain functions.

\subsection*{Multiscale brain networks and schizophrenia}
Schizophrenia is a psychiatric disorder that is characterized by hallucinations, delusions, a loss of initiative, cognitive dysfunction, as well as disruptions of brain connectivity. Since around 100 years ago, pioneering psychiatrists Carl Wernicke (1848-1905) and Eugen Bleuler (1857-1939) first hypothesized a loosening of association fibres in the brain of schizophrenia. Empirical MRI studies have further provided evidence for widespread disruptions of interregional connectivity in schizophrenia patients (55-57), in particular connectivity between hub regions of the brain (57). These alterations of hub connectivity run in parallel with decreased brain network efficiency (58) and impaired rich club organization in schizophrenia (59), suggesting an disrupted whole-brain communication and information integration in schizophrenia.

In terms of microscale neuropathology, a rich body of brain cytoarchitectonic studies has shown alterations of neuronal projections in schizophrenia (60). For example, dendritic spines pathology has been observed in schizophrenia, in particular in layer III pyramidal neurons in prefrontal cortex, indicating disruptions of cortico-cortical and cortico-thalamus excitatory connections in schizophrenia (61). Furthermore, brain regions with more decreased spine density of pyramidal neurons were found to show more connectivity loss at the macroscale (47), confirming the macro-micro linkage in the neuropathology of schizophrenia. 

Genome-wide association studies (GWAS) have shown that schizophrenia is highly polygenic (62), namely, multiple genes with additive small effects lead to the disorder. The recent GWAS on 40,675 schizophrenia patients and 64,643 controls have identified >100 genomic risk loci and >300 risk genes (63). The identified genes were shown to be associated with the formation of microscopic neuronal projection (63). More intriguingly, cortical transcriptions of such schizophrenia-risk genes have been suggested to be associated with the macroscale dysconnectivity patterns in schizophrenia, showing brain regions with higher gene expression to be more dysconnected by white matter tracts (64). Taking together, microscale brain disruptions and macroscale dysconnectivity are tightly linked and together form the complex neuropathology of schizophrenia.

\section*{Outline of the thesis}
The aim of this thesis is to explore the neurobiological mechanisms underlying the organization of human brain connectome and to scrutinize multiscale neuropathology of schizophrenia. In chapter 2, we take advantage of the state-of-the-art BigBrain histological dataset (65) and examine how cortical laminar cytoarchitecture relates to the organization of macroscale brain connectome. In chapter 3, we delve into the organizing principles of resting-state functional networks from the perspective of evolutionary genetics. We examine the expansion of cognitive functional networks in human evolution, and we investigate the transcription of human-accelerated genes in human-expanded cognitive functional networks. What’s more, we present associations between human-accelerated genes and cognitive abilities and risks of psychiatric disorders in today’s population based on recent GWAS results. Extending these results, chapter 4 describes a web-based application, GAMBA, that can be used to explore the association between gene expression and a wide range of brain imaging traits. Three examples are provided to demonstrate the usability of GAMBA, with one example specifically presents the association between genes enriched in supragranular layers of the human cortex and the macroscale hub organization. In chapter 5, we examine alterations of rich-club organization of the macroscale brain connectome and structural-functional coupling in first-episode, medication-naïve schizophrenia patients using two independent dataset. In chapter 6, we use magnetization transfer (MT) imaging data from schizophrenia patients to provide an estimates of microscale neuropathology, such as myelination, and examine how microscale neuropathology correlate to macroscale dysconnectivity in schizophrenia. Finally, chapter 7 summarizes the findings discussed in this thesis and discusses methodological considerations and future directions.

\end{refsection}
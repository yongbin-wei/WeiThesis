\pagestyle{MyStyle}

\chapter[Introduction]{Introduction}
\label{ch:intro}

\vspace{3cm}

\begin{refsection}
\newpage
\section*{The brain connectome}
The brain is the most sophisticated organ in the human body. The adult human brain consists of approximately 100 billion neurons and 150 trillion synapses (Herculano-Houzel, 2012). Through axonal projections and synapses, neurons are wired into neural circuits that interconnect to one another to form the large-scale brain network, namely the connectome (Sporns, 2011). Mapping the microscale connectome of the human brain is extremely challenging due to the huge amount of synapses and the extremely densely packed neuronal circuits, with the largest reconstruction of $ \sim $ 500,000 cubic micrometers of the mouse connectome available at present (Motta et al., 2019). Instead, reconstructing the macroscale human brain connectome formed by brain regions and interregional brain connectivity is way more feasible due to the emergence of non-invasive brain imaging techniques such as magnetic resonance imaging (MRI) (Bullmore and Sporns, 2009; Sporns, 2011). The macroscale brain connectivity usually describes anatomical white matter tracts that are reconstructed using diffusion weighted imaging (DWI) (Hagmann et al., 2007) or the temporal relationship of intrinsic neural activities measured by functional MRI (fMRI) (Friston, 1994). The macroscale brain connectome has been claimed as a highly efficient network that enables both functional specialization and functional integration within the whole brain (Bullmore and Sporns, 2012). It supports a rich body of functional dynamics and interactions that underlie human cognition and behaviors, evidenced by associations between the connectome organization and cognitive processing in humans (van den Heuvel et al., 2009b; Baggio et al., 2015; Barbey, 2018) and brain dysconnectivity in mental disorders (Crossley et al., 2014; Fornito and Bullmore, 2015). Therefore, the macroscale brain connectome provides an invaluable model for understanding the neural basis of human cognitive abilities and the neuropathology of brain disorders.

\section*{Aim of this thesis}
One of the major challenges in the field of connectomics is to understand the wiring principle of the human brain connectome. What shapes brain connectivity and the connectome architecture in the human brain? This thesis aims to address this question by integrating multiscale data, ranging from brain transcriptomics, histology, and multi-modal MRI, to explore the genetic and cytoarchitectonic underpinnings of the human brain connectome. Furthermore, this thesis aims to examine multiscale, multimodal connectome abnormalities in psychiatric disorders, in particular schizophrenia, to provide insights into the neuropathology of the disorder. We expect our integrative approaches to be useful in gaining a better knowledge of the wiring principle of the human brain connectome, and ultimately, to gain a deeper understanding of human cognition, behavior and brain disorders.

\section*{General background}
\subsection*{Topological characteristics of brain network}
The macroscale brain connectome is a complex network composed of comprehensive sets of connections between brain areas. Mathematically, a network, or a graph, is defined as a set of nodes and edges. Concerning the human brain network, nodes are usually defined according to the parcellation of unique brain regions, which are established based on cortical cytoarchitecture (Brodmann and Gary, 2006), gyrification (Desikan et al., 2006), connectivity profiles (Fan et al., 2016), or a combination of multiple modalities (Glasser et al., 2016). Edges are defined as interareal structural and functional connectivity among brain regions. Structural connectivity in the human brain commonly reflects white matter bundles reconstructed using DWI (Hagmann et al., 2008). The size of white matter bundles or the microscopic features such as myelination can be inferred and taken as the weight of structural connectivity. Functional connectivity in the human brain is characterized as the statistical dependency of brain activity between brain regions (Friston, 2011), such as the synchrony of blood oxygenation level-dependent (BOLD) signals derived from resting-state fMRI. Recent studies on the human structural and functional connectome have revealed the most fundamental attribute of the connectome: the capability of promoting segregation and integration of neural information (Sporns, 2013).

Neural information segregation complies with the long-standing argument of brain functional specialization, with evidence from a wide range of anatomical, physiological, and neuroimaging studies showing different brain areas to be specialized for different cognitive domains (Finger, 1994). Connectome analyses add to this by demonstrating the architecture of communities in the functional connectome (Sporns and Betzel, 2016). Brain regions within a community are way more densely connected to each other in contrast to being connected to nodes outside the community. The observed sets of communities coincide with the distributed $``$resting-state networks$"$  (RSNs) that are derived by independent component analysis (ICA) on whole-brain resting-state fMRI data (Beckmann et al., 2005). The RSNs correspond to functional systems in distinct behavioral or cognitive domains (Smith et al., 2009; Laird et al., 2011), ranging from the default mode (DMN) and frontoparietal control networks to the dorsal attention, ventral attention, sensorimotor, and visual networks (Power et al., 2011; Yeo et al., 2011). These spatially-distributed functional systems have been found to be supported by underlying anatomical connections (van den Heuvel et al., 2009a) and to be globally integrated via pivotal $``$hub$"$  regions to facilitates complex brain functions (van den Heuvel and Sporns, 2013a).

Neural information integration describes the integration of neural information among distributed functional systems and is served by a specific set of brain hub regions (van den Heuvel and Sporns, 2013b). Hub regions are mostly located in multimodal association areas, such as the precuneus, insular cortex, superior frontal cortex, temporal cortex, and lateral parietal cortex (van den Heuvel and Sporns, 2013b). Furthermore, these hub regions are not only highly connected to the rest of the brain, but also tightly connected with each other, forming a $``$rich club$"$ \  (van den Heuvel and Sporns, 2011; Griffa and van den Heuvel, 2018). Rich club areas and their connections enable short communication pathways among non-rich-club, peripheral areas, thus promote network efficiency with a trade-off against high wiring cost of connections among distant hub regions (van den Heuvel et al., 2012). These central regions are also flexible over time, indicative of the capability of dynamically switching brain states from more localized processing (i.e., segregation) to more globalized synchronization (i.e., integration) (Zalesky et al., 2014; Shine et al., 2016). Such flexibility enables the brain to adapt to changing demands (Cole et al., 2013) and facilitates faster and more accurate cognitive performance (Shine et al., 2016).

Together, neural information segregation and integration are two fundamental features of the human brain connectome. These two features play a key role in complex human brain dynamics underlying cognition and behavior. One major goal of the current thesis is to understand the origin and emergence of such an organization of the human brain connectome.

\subsection*{Neurobiological basis of brain networks}
In recent years, studies have bridged the microscale cellular and molecular brain structure to the macroscale connectome organization to examine the neurobiological mechanisms of the connectome (Scholtens and van den Heuvel, 2018; van den Heuvel et al., 2019a). Brain regions are known to show a large variety of cellular composition (Brodmann, 1909; von Economo and Koskinas, 1925), for example, larger and more spinous pyramidal cells are observed in associative prefrontal cortex compared to primary visual cortex (Elston et al., 2001; Elston, 2003). Extending this cytoarchitectonic variation, studies have further proposed that the inter-regional cytoarchitectonic differentiation plays a critical role in shaping cortico-cortical connections (Barbas, 2015). Brain regions with similar cytoarchitecture are more likely to be connected both anatomically and functionally (Barbas and Rempel-Clower, 1997; Barbas, 2015; Beul et al., 2015; Beul et al., 2017; Goulas et al., 2017). Moreover, cortical cytoarchitecture is also associated with functional segregation and integration in the brain. Paralimbic regions with an undifferentiated laminar structure, along with absent or disrupted cortical layer IV, are prominently involved in between-network connectivity across functional networks (Wylie et al., 2015). Densely-connected rich-club regions show the largest basal dendritic tree size, spine density, and neuron soma size of layer III pyramidal neurons (Scholtens et al., 2014b; van den Heuvel et al., 2015; van den Heuvel et al., 2016b). These observations suggest a highly complex neuronal structure in rich-club regions, such that rich-club regions have the capacity to deal with the costly neural information integration. 

Extending the association between the macroscale connectome and microscale brain cytoarchitecture, studies have further claimed the connectome organization to be related to the underlying genetic information. By combining gene expression and brain connectivity in the rodent brain, it has been shown that the existence of anatomical connectivity could be predicted by gene expression levels (Wolf et al., 2011), and anatomically-connected regions tended to show more similar expression profiles (French and Pavlidis, 2011). Genes that are mostly correlated with anatomical connectivity are enriched for gene sets involved in neuronal development and axon guidance (French and Pavlidis, 2011). Moreover, the topology of macroscale functional networks in the human brain is related to the co-expression of genes that are enriched for ion channels (Richiardi et al., 2015) and genes that are preferentially expressed in human supragranular cortical layers (Krienen et al., 2016). Rich club regions also show a differentiated level of gene expression coupling for genes regulating the oxidative synthesis and metabolism of ATP, which provide the energetic substrate for neuronal communication (Fulcher and Fornito, 2016). Taken together, these observations suggest a cross-scale association of the microscale transcriptomics and cytoarchitecture, and the macroscale connectome organization.

\subsection*{Multiscale brain networks and schizophrenia}
Schizophrenia is a psychiatric disorder characterized by hallucinations, delusions, a loss of initiative, cognitive dysfunction, as well as disruptions of brain connectivity (Stephan et al., 2009). Since around 100 years ago, pioneering psychiatrists Carl Wernicke (1848-1905) and Eugen Bleuler (1857-1939) hypothesized a loosening of association fibers in the brain of schizophrenia. Empirical MRI studies have provided further evidence for the widespread disruptions of interregional brain connectivity in schizophrenia (Ellison-Wright and Bullmore, 2009; Fitzsimmons et al., 2013; Klauser et al., 2017), in particular disrupted connectivity between hub/rich-club regions of the brain (van den Heuvel et al., 2013; Klauser et al., 2017). These alterations of hub/rich-club connectivity parallel with decreased brain network efficiency (Zalesky et al., 2011), suggesting the disrupted neural information integration in schizophrenia.

In terms of microscale neuropathology, a rich body of brain cytoarchitectonic studies has shown alterations of neuronal projections in schizophrenia (Bakhshi and Chance, 2015). For example, dendritic spines pathology has been observed in schizophrenia, in particular for layer III pyramidal neurons in prefrontal cortex, indicating disruptions of cortico-cortical and cortico-thalamus excitatory connections in schizophrenia (Glausier and Lewis, 2013). Furthermore, a cross-scale study has revealed that brain regions with a more decreased spine density of pyramidal neurons showed more connectivity disruptions at the macroscale (van den Heuvel et al., 2016b), confirming the macro-micro linkage in the neuropathology of schizophrenia. 

Moreover, genome-wide association studies (GWAS) have shown that schizophrenia is a highly polygenic disorder (Schizophrenia Working Group of the Psychiatric Genomics, 2014), namely, multiple genes with additive small effects lead to this disorder. The recent GWAS on 40,675 schizophrenia patients and 64,643 controls have identified >100 genomic risk loci and >300 risk genes (Pardinas et al., 2018). The identified genes have been shown to be associated with the formation of microscopic neuronal projection (Pardinas et al., 2018). More intriguingly, cortical transcriptions of schizophrenia-risk genes have been noted to be associated with the macroscale disconnectivity pattern in schizophrenia, showing that brain regions with higher gene expression were more disconnected at the macroscale (Romme et al., 2017). Taken together, the microscale abnormalities and macroscale disconnectivity are tightly linked in schizophrenia, and they contribute to the complex etiology of schizophrenia.

\section*{Outline of the thesis}
This thesis aims to explore the neurobiological mechanisms underlying the human brain connectome and to scrutinize the multiscale neuropathology of schizophrenia. In \textbf{chapter 2}, I use the state-of-the-art BigBrain histological dataset (Amunts et al., 2013) and examine how cortical laminar cytoarchitecture relates to the organization of the macroscale brain connectome. In \textbf{chapter 3}, I delve into the organizing principles of functional networks from the perspective of evolutionary genetics. I examine the expansion of cognitive functional networks in human evolution, and I investigate gene expression levels of human-accelerated genes (HAR genes) in those human-expanded cognitive functional networks. In addition, I present associations between human-accelerated genes and cognitive abilities and risks of psychiatric disorders in today’s population based on recent GWAS results. In \textbf{chapter 4}, I focus on schizophrenia and examine alterations of rich-club organization of the macroscale brain connectome and structural-functional connectivity coupling in first-episode, medication-naïve schizophrenia patients. In \textbf{chapter 5}, I use magnetization transfer (MT) imaging data to provide an estimation of the microscale neuropathology in schizophrenia, and examine how microscale neuropathology correlate to macroscale dysconnectivity in schizophrenia. Finally, \textbf{chapter 6} summarizes the findings in this thesis and discusses methodological considerations and future directions.

\end{refsection}